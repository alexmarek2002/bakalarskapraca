\documentclass[12pt]{article}
\usepackage{amsmath}

\newcommand{\Prize}[2]{P(#1,#2)}

\title{Bachelor Thesis}
\author{Alex Marek}
\date{\today}

\begin{document}
	
	\maketitle
	
	\section*{LED trees}
	LED (leaves of equal depth)
	
	\maketitle
	
	\section*{relaxed LED trees}
	
	\maketitle
	
	\section*{relaxed LED tree with 3 leaves}
	
	Let T be a tree that contains exactly three vertices \( A \), \( B \), and \( C \) of degree one (i.e. leaves) and exactly one vertex \( S \) of degree tree. Vertex \( S \) lies inside the triangle(convex hull) determined by determined by vertices \( A , B , C \) and is adjacent to all of them. Let us place a vertex \( R \) on an arbitrary one of these edges. This vertex will be referred to as the \textbf{root} of the tree.
	
	For every edge that connects two points, we assign two values:
	\begin{itemize}
	\item \( |UV| \): the Euclidean distance between the two points \( u \) and \( v \),
	\item $\Prize{U}{V}$: the relaxed (or curved) value of the edge between \( u \) and \( v \),
	\end{itemize}
	where \( |UV| \leq  $ \Prize{U}{V}$ \).\\
 	
	A tree is considered a LED tree if $\Prize{}{}$ is equal for all paths from the root to each leaf. In case of three T:
	\begin{align*}
		\Prize{R}{A} = \Prize{R}{B} = \Prize{R}{C}
	\end{align*}
	
	We define the total length \( L() \) of the tree as sum of all $\Prize{}{}$ .
	
	For tree \( T \), we have three options for placing the root vertex \( R \), depending on which edge it lies on. Let us consider the case where \( R \in \overline{CS} \). For this topology the total length would be:
	 \begin{align*}
	 	L(T) = \Prize{A}{S} + \Prize{B}{S} + \Prize{C}{R} + \Prize{R}{S}
	 \end{align*}
	 
	 
	 Now we will look at how minimal Total length depends on certain properties (lenghts of segments): 
	\begin{enumerate}
	\item \( |AS| > |BS|, |AS|+|SR| > |CR| \)
	\item \( |AS| < |BS|, |BS|+|SR| > |CR| \)
	\item \( |AS| > |BS|, |AS|+|SR| \leq |CR| \)
	\item \( |AS| < |BS|, |BS|+|SR| \leq |CR| \)
	\end{enumerate}
	
	
	Let us analyze the first case where \( |AS| > |BS|, |AS|+|SR| > |CR| \).
	We know that:
	\begin{align*}
	\Prize{R}{C} &= \Prize{R}{A} = \Prize{R}{B} \\
	\Prize{R}{C} &= \Prize{R}{S} + \Prize{S}{A} = \Prize{R}{S} + \Prize{S}{B} \\
	&\Rightarrow \Prize{S}{A} = \Prize{S}{B}
	\end{align*}
	We also know that \( |UV| \leq  $ \Prize{U}{V}$ \), this means that the smallest possible value for \Prize{S}{A} is \(|SA|\) and for \Prize{S}{R} is \(|SR|\).
	\begin{align*}	
		\Prize{S}{A} = \Prize{S}{B} &= |SA| \\
		\Prize{S}{R} &= |SR| \\
		\Prize{R}{C} = \Prize{R}{S} + \Prize{S}{A} &= |SA| + |SR|
	\end{align*}
	The total length \( L() \) of three T is now computed as follows:
	\begin{align*}	
		L(T) &= \Prize{S}{A} + \Prize{S}{B} + \Prize{C}{R} + \Prize{R}{S} \\
		&= |SA| + |SA| + |SR| + |SA| + |SR| \\
		&= 2 |SR| + 3 |SA|
	\end{align*}
	
	
	For second case \( |AS| < |BS|, |BS|+|SR| > |CR| \), similarly as in first case:
	\begin{align*}
		\Prize{S}{B} = \Prize{S}{A} &= |SB| \\
		\Prize{S}{R} &= |SR| \\
		\Prize{R}{C} &= |SB| + |SR|
	\end{align*}
		and total length \( L() \) of three T is:
	\begin{align*}	
		L(T) = 2 |SR| + 3 |SB|
	\end{align*}
	
	Let us analyze the third case where \( |AS| > |BS|, |AS|+|SR| \leq |CR| \).
	We know that:
	\begin{align*}
		\Prize{S}{A} = \Prize{S}{B}
	\end{align*}
	also that smallest possible value for \Prize{S}{A} is \(|SA|\).
	\begin{align*}	
		\Prize{S}{A} = \Prize{S}{B} = |SA| \\
	\end{align*}
	since \(|CR| \geq |AS|+|SR| > |BS|+|SR| \), the smallest possible value for \Prize{R}{C} is \(|CR|\)
	\begin{align*}
	 	\Prize{R}{C} &= \Prize{R}{A} = \Prize{R}{B} = |CR| \\
	 	\Prize{R}{C} &= \Prize{R}{S} + \Prize{S}{A} \\
	 	\Prize{R}{S} &= |CR| - |AS|
	\end{align*}
	The total length \( L() \) of three T is then computed as follows:
	\begin{align*}	
		L(T) &= \Prize{S}{A} + \Prize{S}{B} + \Prize{C}{R} + \Prize{R}{S} \\
		&= |SA| + |SA| + |CR| + |CR| - |SA| \\
		&= |SA| + 2 |CR|
	\end{align*}

	For fourth case \( |AS| < |BS|, |BS|+|SR| \leq |CR| \), similarly as in third case:
	\begin{align*}
		\Prize{S}{B} &= \Prize{S}{A} = |BS| \\
		\Prize{R}{C} &= \Prize{R}{A} = \Prize{R}{B} = |CR| \\
	 	\Prize{R}{S} &= |CR| - |BS|
	\end{align*}
	and total length \( L() \) of three T is:
	\begin{align*}	
		L(T) = |SB| + 2 |CR|
	\end{align*}
		 
		 
	We have not yet considered the case where \( |AS| = |BS| \). Let us take a closer look at it. 
	
	First, we assume that \( |AS| + |SR| > |CR| \) and \( |BS| + |SR| > |CR| \), and let \( |SR| \) be a fixed distance.
	\begin{align*}
		|AS| = |BS| = d, \quad d \in R \\
		L(T_1) = 2|SR| + 3|SA| = 2|SR| + 3|BS| = 2|SR| + 3d
	\end{align*}
	Now suppose that \( |AS| = d + k \), where \( k > 0 \). Then the total length becomes:
	\[
	L(T_2) = 2|SR| + 3(d + k) = 2|SR| + 3d + 3k > L(T_1)
	\]
	Similarly, if \( |BS| = d + k \), we have:
	\[
	L(T_3) = 2|SR| + 3(d + k) = 2|SR| + 3d + 3k > L(T_1)
	\]
	In both cases, when either \( |AS| \) or \( |BS| \) deviates from \( d \), the total length of the tree increases compared to the symmetric case. It follows that the minimal total length is achieved when \( |AS| = |BS| \).
	
	Now consider the second case, where \( |AS| + |SR| \leq |CR| \) and \( |BS| + |SR| \leq |CR| \), and let \( |CR| \) be fixed:
	\begin{align*}
		|AS| = |BS| = d, \quad d \in R \\
		L(T_1) = 2|CR| + |SA| = 2|CR| + |SB| = 2|CR| + d
	\end{align*}
	Now suppose that \( |AS| = d + k \), where \( k > 0 \). Then the total length becomes:
	\[
	L(T_2) = 2|CR| + d + k > L(T_1)
	\]
	If \( |BS| = d + k \), we have:
	\[
	L(T_3) = 2|CR| + d + k > L(T_1)
	\]
	Again, it is clear that the minimal total length is achieved when \( |AS| = |BS| \).\\

	Now let us analyze the case when \(|AS|+|SR| = |CR| \).
	The total length would be \[ L(T) = |SA| + 2 |CR| = L(T) = |AS| + 2|AS| + 2|SR| = 3|AS| + 2|SR|  )\]
	Let \( d > 0 \) and suppose we fix \( |CR| \) and \( |SR| \), and increase \( |AS| \) by \( d \):
	\[
	|AS|^* = |AS| + d \quad \Rightarrow \quad L(T_1) = 2|SR| + 3|AS|^* = 2|SR| + 3(|AS| + d) = L(T) + 3d > L(T)
	\]
	
	Now fix \( |CR| \) and \( |AS| \), and increase \( |SR| \) by \( d \):
	\[
	|SR|^* = |SR| + d \quad \Rightarrow \quad L(T_2) = 2|SR|^* + 3|AS| = 2(|SR| + d) + 3|AS| = L(T) + 2d > L(T)
	\]
	
	Now fix \( |SR| \) and \( |AS| \), and increase \( |CR| \) by \( d \):
	\[
	|CR|^* = |CR| + d \quad \Rightarrow \quad L(T_3) = |AS| + 2|CR|^* = |AS| + 2(|CR| + d) = L(T) + 2d > L(T)
	\]
	
	In all cases, any deviation from the equality \( |AS| + |SR| = |CR| \) leads to an increase in the total length.  
	Therefore, the minimal total length of the tree is achieved exactly when
	\[
	|AS| + |SR| = |CR|
	\]
	A completely analogous argument applies when \( |BS| + |SR| = |CR| \).  
	Again, increasing \( |BS| \), \( |SR| \), or \( |CR| \) individually while keeping the others fixed leads to an increase in the total length.  
	Hence, the minimal total length is achieved exactly when:
	\[
	|BS| + |SR| = |CR|
	\]
	
	
	This leads us to the question of how to place the root \( R \in \overline{CS} \) for given points \( A, B, C, S \), so that the total length of the LED tree is minimized. From the constraint \( |AS|+|SR| = |CR|  \), we derive:
	
	\begin{align*}
		|AS| + |CS| &= |AS| + |SR| + |RC| \\
		|AS|+|CS| &= |AS|+|SR|+|AS|+|SR| \\ 
		|CS| &= |AS|+2|SR| \\
		|SR| &= \frac{|CS| - |AS|}{2} \\
		\Rightarrow \quad |RC| &= |CS| - |SR| = |CS| - \frac{|CS| + |AS|}{2}
	\end{align*}
	Therefore, to minimize the total length of the LED tree, the root \( R \) should be placed on edge \( \overline{CS} \) such that its distance from \( S, C\) is:
	\[
	|SR| = \frac{|CS| - |AS|}{2},  |RC| = |CS| - \frac{|CS| + |AS|}{2}
	\]
	
	\end{document}
