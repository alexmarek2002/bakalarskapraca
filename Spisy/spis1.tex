\documentclass[12pt]{article}
\usepackage{amsmath}

\title{Bachelor Thesis}
\author{Alex Marek}
\date{\today}

\begin{document}
	
	\maketitle
	
	\section*{LED trees}
	LED (leaves of equal depth)
	
	\maketitle
	
	\section*{relaxed LED tree with 3 leaves}
	
	Let \( T \) be a tree that contains exactly three vertices \( A \), \( B \), and \( C \) of degree one (i.e. leaves) and exactly one vertex \( S \) of degree tree.
	The triangle determined by vertices \( A , B , C \) is acute-angled and vertex \( S \) is inside this triangle and is adjacent to all the leaves. 
	Lets call these edges as follows:
	\[
	\alpha = |AS| \quad \beta = |BS| \quad \gamma = |CS|
	\]
		

	Now lets place a vertex \( R \) on an arbitrary one of these edges. This vertex will be referred to as the \textbf{root} of the tree.

	// niekde vysvetlit relaxed co znamena
	
	For every edge that connects two points, we assign two values:
	\begin{itemize}
	\item \( D_f(u, v) \): the Euclidean distance between the two points \( u \) and \( v \),
	\item \( D_j(u, v) \): the relaxed (or curved) value of the edge between \( u \) and \( v \),
	\end{itemize}
	where \( D_j(u, v) \geq D_f(u, v) \).
 	
	A tree is considered a LED tree if \( D_j \) is equal for all paths from the root to each leaf.
	\[
	D_j(R, A) = D_j(R, B) = D_j(R, C)
	\]
	
	We define the total length \( L() \) of the tree as sum of all \( D_j \). //este neni dobre naformulovane 
	
	This section is devoted to analyze the properties of a minimal LED tree with exactly three leaves.
 	Thus where \( L \) for set of leaves is minimal.
 	
	On tree \( T \), we have three options for placing the root vertex \( R \), depending on which edge it lies on. Let us consider the case where \( R \in \gamma \). For this topology the \( L(T) = D_j(A, S) + D_j(B, S) + D_j(C, R) + D_j(R, S) \). Minimal L(T) depends on position of \( S \), to be more specific, in which segment of the triangle \( A , B , C \), \(S\) lies. We can determine four segments of triangle that are specified by inequalities:
	\begin{enumerate}
	\item \( \alpha > \beta, \quad \alpha < \gamma \)
	\item \( \beta > \alpha, \quad \beta < \gamma \)
	\item \( \alpha > \beta, \quad \alpha > \gamma \)
	\item \(  \beta > \alpha, \quad \beta > \gamma \)
	\end{enumerate}
	
	Lets further analyse the first segment:
	\begin{align*}
		D_j(R, C) = D_j(R, S) + D_j(S, A) = D_j(R, S) + D_j(S, B) \\
		\Rightarrow \quad D_j(S, A) = D_j(S, B) \\
		D_f(S, A) > D_f(S, B) \\
		min(D_j(S,A)) = min(D_j(S,B)) = D_f(S, A) \\
	\end{align*}
	Since our goal si to minimize the three lets consider \( D_j(S, A) = D_j(S, B) = D_f(S, A) \) and also since we do not have any condition for \(D_j(R, S)\) and \(min(D_j(R, S)) = D_f(R, S)\), lets consider \( D_j(R, S) = D_f(R, S)  \).
	The total length \( L() \) of three T is now computed as follows:
	\begin{align*}
		L(T) &= D_j(R, C) + D_j(R, S) + D_j(S, A) + D_j(S, B)\\
		&= D_f(R, S) + D_f(S, A) + D_f(R,S) + D_f(S, A) + D_f(S, A)\\
		&= 2 D_f(R, S) + 3 D_f(S, A) \\
	\end{align*}
	
	similarly for second segment: 
	...
	\begin{align*}
		L(T) = 2 D_f(R, S) + 3 D_f(S, B)
	\end{align*}
	 // chyba podmienka aby \(D_j(R, B/A) > D_j(R, C) \)
	 // odtialto dole je to volnopis
	 
	 For third segment, \(D_j(R, B/A) > D_j(R, C) \) can not happen since \(\alpha > \gamma \)
	for that reason we will get the shortest tree by minimizing \( D_f(R,S) \) ... \(R = S\). 
	This means that the length \( L() \) of three is:
	\begin{align*}
		L(T) = 3 D_f(S, A)
	\end{align*}
	
	Fourth Segment:
	\begin{align*}
		L(T) = 3 D_f(S, B)
	\end{align*}
		 
	 
		 
\end{document}
